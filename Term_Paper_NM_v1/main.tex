\documentclass{article}
\usepackage[utf8]{inputenc}
\PassOptionsToPackage{hyphens}{url}\usepackage{hyperref}
\setlength{\parindent}{4em}
\setlength{\parskip}{1em}

\title{\textbf{Science of Nuclear Medicine - Update1}}
\author{Aditi Bhattacharya}
\date{21 January 2022}

\begin{document}

\maketitle
\section*{Abstract}
Nuclear medicine is the use of radionuclides in medicine for diagnosis, staging of disease, therapy and monitoring the response of a disease process. It is also a powerful translational tool in the basic sciences, such as biology, in drug discovery and in pre-clinical medicine. Developments in nuclear medicine are driven by advances in this multidisciplinary science that includes physics,
chemistry, computing, mathematics, pharmacology and biology. This paper is written with an aim to understand all the multidisciplinary aspects of nuclear medicine, with a prospect to provide a valuable hypothesis which can enhance this field.
\end{document}
